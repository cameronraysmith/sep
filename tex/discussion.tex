%!TEX root = ../plos_template.tex
In biology, prevailing wisdom suggests that anything is possible unless it has been demonstrated to some reasonable degree of confidence to be physically impossible. But the logical implications of what is known to be physically impossible are rarely brought to bare with full force on biological problems. This process can be moved forward by taking counterfactual statements and abductive reasoning very seriously. For example, a first step toward investigating the apparent absence of reverse flow of information from phenotype to genotype should be to consider whether the imposition of any such constraint could logically result from natural selection or any other facet of the evolutionary process. In this light and to contribute to the broader goal of establishing an integrated framework that synthesizes hypothesized intrinsic and extrinsic constraints necessary to understand the functioning and evolution of biological systems, here we have attempted to trace a path from a collection of mathematical facts relating spaces of probability distributions to their potential impact upon models of gene expression and, by the obvious extension, to evolutionary processes.

One goal of measuring gene expression at transcriptomic scale has to be to uncover the structure of the generative process encoded in the GRN interactions involved. This fact has motivated computational biologists to develop a large collection of algorithms to infer aspects of this structure \cite{DeSmet2010} and experimental biologists to compare networks on the basis of their hierarchical and modular architecture \cite{Ideker2012}. Our model and its framework puts forward a fundamental constraint, that while not unique to gene regulatory networks, must nevertheless impact their structure at the level of abstraction described here. The fact that GRNA impacts the structure of the entire space of probability distributions over genotype-phenotype maps---and not just a parameter-dependent fragment of it---that could possibly be achieved given that architecture provides fodder for natural selection to mold GRNA in a more precise manner than has previously been appreciated.

We suspect that there are many other simple physical and mathematical constraints that can be brought to bear on biological problems. At first, statements of such may only be effective at a relatively high level of abstraction. However, as a larger number of broad constraints are synthesized, they can be applied to increasingly detailed models that integrate omic scale information over developmental and evolutionary timescales. Approaches of the type we have exhibited here hold promise for cashing out the tremendous potential that continues to be built up in databases of high-throughput genomic data.
