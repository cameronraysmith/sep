%!TEX root = ../plos_template.tex
It is clear that in most cases, when biological networks are studied, we remove a subnetwork from a larger context~\cite{Alon2007}. The structure of the network context plays a crucial role in determining whether or not unsatisfiable constraints on the stochastic dynamical patterns of network states may arise at all. Our results suggest that mutually incompatible constraints are only capable of arising when the network architecture contains a cycle. Moreover, the likelihood of mutually incompatible constraints arising relative to network architecture increases with the number of cycles in that network architecture. An evolutionary process exhibiting uniform sampling over the space of network architectures and the space of possible constraints within each network architecture, would thus be expected to exhibit a bias toward the breakage of cycles. It will be important in future work to examine this prediction more closely in the context of developing bottom-up stochastic process models that allow for the explicit encoding and solution of models of more complex biological networks~\cite{Walczak2009,Mugler2009}. It is possible that the specific dynamics of a given network context may lead to apparent access to correlations that are otherwise inaccessible. In the case of gene-regulatory networks, this may occur via a form of cis-regulation that enables the breakage of statistical dependence in a time-dependent manner \ref{fig:condgenescenario}. But such a scenario seems much less plausible than the ability to resolve inconsistency by breaking cycles, for example via gene duplication. In the long term, the latter corresponds to what is observed in hierarchically organized transcription factor networks \cite{Jothi2009,Bhardwaj2010,Chalancon2012,Colm}. The mechanism outlined here is consistent with previous analyses of hierarchical modular gene regulatory network architectures~\cite{Ravasz2002,Segre2005,Wagner2007,Erwin2009,Jothi2009,Bhardwaj2010,Colm}.

To contribute to the broader goal of establishing an integrated framework that synthesizes hypothesized intrinsic and extrinsic constraints necessary to understand the functioning and evolution of biological systems, here we have traced a path from biological network architecture to network state constraint satisfiability, and, via the impact of network states on higher-level properties culminating in macroscopically observable phenotypes, to evolutionary processes. In the particular context of gene-regulatory networks, one goal of measuring gene expression at transcriptomic scale is to uncover the structure of the generative process encoded in the interactions involved, but, so far, even the most sophisticated methods of describing them at the mechanistic level are only solvable for extremely simple regulatory network architectures~\cite{Walczak2009,Mugler2009}. This fact has, in part, motivated computational biologists to develop a large collection of algorithms to infer aspects of this structure~\cite{Anastassiou2007,DeSmet2010} and experimental biologists to compare networks on the basis of their hierarchical and modular architecture~\cite{Ideker2012}. Our model and its framework put forward a class of fundamental constraints that may impact the expected structure of biological networks. The fact that the satisfiability of the space of possible constraints that can be imposed upon a network is dependent upon the structure of the network context provides a mechanism by which natural selection may exhibit a fundamental bias in its sampling of biological network architectures.
% In biology, as in other sciences such as physics~\cite{Gell-Mann1956}, prevailing wisdom suggests that anything is allowed unless it has been demonstrated to some reasonable degree of confidence to be forbidden. We suspect that there are many other simple physical and mathematical constraints that can be brought to bear on biological problems. Constraint-based models that extract invariants of metabolic and other biochemical reaction networks provide a shining example of this approach~\cite{Karp2012,Bordbar2014}. The use of fundamental constraints to characterize geometric properties of phenotype space have begun to elicit applications of this approach to evolutionary processes~\cite{Shoval2012,Sheftel2013,Jordan2013}. At first, statements of such may only be effective at a relatively high level of abstraction. However, as a larger number of broad constraints are organized and integrated into a common language they can be applied to increasingly detailed models that integrate omic scale information over developmental and evolutionary timescales~\cite{Gunawardena2013}. Approaches of the type we have exhibited here hold promise for realizing the tremendous potential that continues to be built up in databases of high-throughput genomic data.

% The ultimate goal of constraint-based approaches toward understanding molecular interaction networks should be to develop a theory capable of expressing the manner in which information is stored and transmitted in networks of molecular interactions~\cite{Tkacik2011a,Bialek2012} and how such processes produce the higher-order properties such as phenotypes that serve as substrate for natural selection~\cite{Mora2013}. Subtle but essential developments in information theory and its applications makes connections between molecular networks~\cite{Cheong2011,Brennan2012} and networks of neurons immediate~\cite{Stanley1999,Balduzzi2008,Balduzzi2009,Balduzzi2012}, while also supporting generalization from molecular networks to organisms and their populations in application to evolutionary processes~\cite{Kussell2005a,Rivoire2011}. Common to all of these instantiations is the drive toward understanding the manner in which relationships between some interacting entities and the mutual constraints implicit to such interactions can be represented mathematically in both algebraic and geometric form. While this study has provided another instantiation of such an approach, it seems to be a mere fragment of a deeper theory requiring expression in terms of a language that curious specialists from each of the relevant fields can all understand well enough to speak.
