%!TEX root = ../plos_template.tex
\subsection*{Constraint deduction method to derive H-representation}
\ref{eq:eqmat} shows the equalities derived by computing the cokernel of the matrix given in Table \ref{tab:logmat222} and adjoining rows that enforce the normalization of the marginal distributions.
\begin{equation}\label{eq:eqmat}
\begin{aligned}
\begin{bmatrix}
  -1 & -1 & 0 & 0 & 1 & 1 & 0 & 0 & 0 & 0 & 0 & 0 & 0 & 0 & 0 & 0 & 0\\
  0 & 0 & -1 & -1 & 0 & 0 & 1 & 1 & 0 & 0 & 0 & 0 & 0 & 0 & 0 & 0 & 0\\
  -1 & 0 & -1 & 0 & 0 & 0 & 0 & 0 & 1 & 1 & 0 & 0 & 0 & 0 & 0 & 0 & 0\\
  0 & -1 & 0 & -1 & 0 & 0 & 0 & 0 & 0 & 0 & 1 & 1 & 0 & 0 & 0 & 0 & 0\\
  0 & 0 & 0 & 0 & 0 & 0 & 0 & 0 & -1 & 0 & -1 & 0 & 1 & 1 & 0 & 0 & 0\\
  0 & 0 & 0 & 0 & -1 & 0 & -1 & 0 & 0 & 0 & 0 & 0 & 1 & 0 & 1 & 0 & 0\\
  -1 & -1 & -1 & -1 & 1 & 0 & 1 & 0 & 1 & 0 & 1 & 0 & -1 & 0 & 0 & 1 & 0\\
  1 & 1 & 1 & 1 & 0 & 0 & 0 & 0 & 0 & 0 & 0 & 0 & 0 & 0 & 0 & 0 & 1\\
  0 & 0 & 0 & 0 & 1 & 1 & 1 & 1 & 0 & 0 & 0 & 0 & 0 & 0 & 0 & 0 & 1\\
  0 & 0 & 0 & 0 & 0 & 0 & 0 & 0 & 1 & 1 & 1 & 1 & 0 & 0 & 0 & 0 & 1\\
  0 & 0 & 0 & 0 & 0 & 0 & 0 & 0 & 0 & 0 & 0 & 0 & 1 & 1 & 1 & 1 & 1\\
\end{bmatrix}
\end{aligned}
\end{equation}
The final column represents the right-hand side of each equality. It turns out all but one of the normalization conditions is linearly dependent with respect to the other equalities and so we can reduce this set of $7+4=11$ constraints to $8$
\begin{equation}\label{eq:kceqsrefa}
\begin{aligned}
\begin{bmatrix}
  1 & 0 & 0 & -1 & 0 & 0 & 1 & 1 & 0 & 0 & 1 & 1 & 0 & 0 & 0 & 0 & 1\\
  0 & 1 & 0 & 1 & 0 & 0 & 0 & 0 & 0 & 0 & -1 & -1 & 0 & 0 & 0 & 0 & 0\\
  0 & 0 & 1 & 1 & 0 & 0 & -1 & -1 & 0 & 0 & 0 & 0 & 0 & 0 & 0 & 0 & 0\\
  0 & 0 & 0 & 0 & 1 & 0 & 1 & 0 & 0 & 0 & 0 & 0 & 0 & 1 & 0 & 1 & 1\\
  0 & 0 & 0 & 0 & 0 & 1 & 0 & 1 & 0 & 0 & 0 & 0 & 0 & -1 & 0 & -1 & 0\\
  0 & 0 & 0 & 0 & 0 & 0 & 0 & 0 & 1 & 0 & 1 & 0 & 0 & 0 & 1 & 1 & 1\\
  0 & 0 & 0 & 0 & 0 & 0 & 0 & 0 & 0 & 1 & 0 & 1 & 0 & 0 & -1 & -1 & 0\\
  0 & 0 & 0 & 0 & 0 & 0 & 0 & 0 & 0 & 0 & 0 & 0 & 1 & 1 & 1 & 1 & 1\\
\end{bmatrix}
\end{aligned}
\end{equation}
These equalities can now be substituted into the positivity inequalities necessary to define any space of probability distributions. This yields a set of inequalities \ref{eq:polrepindineqs} that specify an H-representation of a polytope. This is the modular polytope, which is a subspace of $\Delta_3^{\otimes 4}$ associated to distributions consistent with the linear transformation $\mathbf{G}$
\begin{equation}\label{eq:polrepindineqs}
\begin{aligned}
\begin{bmatrix}
  1 & 1 & -1 & -1 & -1 & -1 & 0 & 0 & 0\\
  0 & -1 & 0 & 0 & 1 & 1 & 0 & 0 & 0\\
  0 & -1 & 1 & 1 & 0 & 0 & 0 & 0 & 0\\
  1 & 0 & -1 & 0 & 0 & 0 & -1 & 0 & -1\\
  0 & 0 & 0 & -1 & 0 & 0 & 1 & 0 & 1\\
  1 & 0 & 0 & 0 & -1 & 0 & 0 & -1 & -1\\
  0 & 0 & 0 & 0 & 0 & -1 & 0 & 1 & 1\\
  1 & 0 & 0 & 0 & 0 & 0 & -1 & -1 & -1\\
  0 & 1 & 0 & 0 & 0 & 0 & 0 & 0 & 0\\
  0 & 0 & 1 & 0 & 0 & 0 & 0 & 0 & 0\\
  0 & 0 & 0 & 1 & 0 & 0 & 0 & 0 & 0\\
  0 & 0 & 0 & 0 & 1 & 0 & 0 & 0 & 0\\
  0 & 0 & 0 & 0 & 0 & 1 & 0 & 0 & 0\\
  0 & 0 & 0 & 0 & 0 & 0 & 1 & 0 & 0\\
  0 & 0 & 0 & 0 & 0 & 0 & 0 & 1 & 0\\
  0 & 0 & 0 & 0 & 0 & 0 & 0 & 0 & 1\\
\end{bmatrix}
\end{aligned}
\end{equation}
A row $(a_0,a_1,...,a_d)$ corresponds to the inequality $a_0 + a_1 x_1 + ... + a_d x_d >= 0$. The embedded identity matrix has, in this particular case eight, rows that specify the positivity of the variables corresponding to each of the, in this particular case eight, dimensions. Transforming this inequality or H-representation to a vertex or V-representation of the modular polytope produces \ref{eq:vrepfromhrep}.
\begin{equation}\label{eq:vrepfromhrep}
\begin{aligned}
\begin{bmatrix}
  1 & 0 & 0 & 0 & 0 & 0 & 0 & 0 & 1\\
  1 & 0 & 0 & 0 & 0 & 0 & 0 & 1 & 0\\
  1 & 0 & 0 & 0 & 0 & 0 & 1 & 0 & 0\\
  1 & 1/2 & 1/2 & 0 & 1/2 & 0 & 1/2 & 1/2 & 0\\
  1 & 1/2 & 0 & 1/2 & 0 & 1/2 & 1/2 & 1/2 & 0\\
  1 & 0 & 0 & 0 & 0 & 0 & 0 & 0 & 0\\
  1 & 1/2 & 1/2 & 0 & 0 & 1/2 & 0 & 0 & 1/2\\
  1 & 1/2 & 0 & 1/2 & 1/2 & 0 & 0 & 0 & 1/2\\
  1 & 0 & 0 & 1/2 & 0 & 1/2 & 0 & 0 & 1/2\\
  1 & 0 & 1/2 & 0 & 1/2 & 0 & 0 & 0 & 1/2\\
  1 & 0 & 1/2 & 0 & 0 & 1/2 & 1/2 & 1/2 & 0\\
  1 & 0 & 0 & 1/2 & 1/2 & 0 & 1/2 & 1/2 & 0\\
  1 & 1 & 1 & 0 & 1 & 0 & 0 & 0 & 0\\
  1 & 0 & 0 & 0 & 1 & 0 & 0 & 0 & 0\\
  1 & 0 & 1 & 0 & 0 & 0 & 0 & 0 & 0\\
  1 & 0 & 0 & 1 & 0 & 0 & 1 & 0 & 0\\
  1 & 0 & 0 & 0 & 0 & 1 & 0 & 1 & 0\\
  1 & 0 & 0 & 0 & 1 & 0 & 1 & 0 & 0\\
  1 & 0 & 1 & 0 & 0 & 0 & 0 & 1 & 0\\
  1 & 1 & 0 & 1 & 0 & 1 & 0 & 0 & 1\\
  1 & 1 & 0 & 1 & 1 & 0 & 1 & 0 & 0\\
  1 & 1 & 1 & 0 & 0 & 1 & 0 & 1 & 0\\
  1 & 0 & 0 & 1 & 0 & 0 & 0 & 0 & 1\\
  1 & 0 & 0 & 0 & 0 & 1 & 0 & 0 & 1\\
\end{bmatrix}
\end{aligned}
\end{equation}
