%!TEX root = ../plos_template.tex
\begin{table}[!ht]
\centering
\begin{subtable}[t]{0.4\textwidth}
\centering
\begin{tabular}{ l || c | c | c | r }
	          &	00 & 01 & 10 & 11\\ \hline
    $l_1 l_2$ & $e_1$ & $e_2$ & $e_3$ & $e_4$\\ \hline
    $l_1 l_4$ & $e_5$ & $e_6$ & $e_7$ & $e_8$\\ \hline
    $l_3 l_2$ & $e_9$ & $e_{10}$ & $e_{11}$ & $e_{12}$\\ \hline
    $l_3 l_4$ & $e_{13}$ & $e_{14}$ & $e_{15}$ & $e_{16}$\\
    \hline
    \end{tabular}
    \caption{\gnpm{}}
    \label{tab:gpm}
\end{subtable}
~~~~~~
\begin{subtable}[t]{0.4\textwidth}
\centering
	\begin{tabular}{ l || c | c | c | r }
	          &	00 & 01 & 10 & 11\\ \hline
    $l_1 l_2$ & $p_1$ & $p_2$ & $p_3$ & $p_4$\\ \hline
    $l_1 l_4$ & $p_5$ & $p_6$ & $p_7$ & $p_8$\\ \hline
    $l_3 l_2$ & $p_9$ & $p_{10}$ & $p_{11}$ & $p_{12}$\\ \hline
    $l_3 l_4$ & $p_{13}$ & $p_{14}$ & $p_{15}$ & $p_{16}$\\
    \hline
	\end{tabular}
	\caption{probabilities}
    \label{tab:probabilities}
\end{subtable}
\caption{\gnpm{} and associated probability tables for the two locus-two phenotype value example. We use an abbreviated notation in which we use the rewrite rules $l_1 l_2 \rightarrow \{l_1, l_2\}$ and $00 \rightarrow \{0, 0\}$ and all others analogous to avoid an excessive number of brackets.}
\label{tab:example}
\end{table}
