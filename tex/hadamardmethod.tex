\subsection*{Vertex enumeration method to derive V-representation}
We, therefore, follow Ara\'{u}jo {\it et al.} \cite{Araujo2012} in transforming from probability to expectation value coordinates which allow us to express the coordinates of the vertices of the polytopes corresponding to the joint distributions on non-modular genotype-phenotype mappings, $\Delta_{15}$, and the marginal distributions on modular genotype-phenotype mappings, $\Delta_3^{\otimes 4}$, in a manner that intrinsically enforces normalization and Kolmogorov consistency. Positivity of the probability coordinates can then be enforced via inequalities that result from the transformation from probability to expected value coordinates.

The transformation employs the so-called Hadamard matrix. The Hadamard matrix $H$ of order $n$ has entries in the set $\{+1,-1\}$ and is defined recursively such that all rows have zero inner product and are thus orthogonal:
\begin{equation}
\begin{aligned}\label{eq:hadamard}
H_1 &= [1]\\
H_2 &= \begin{bmatrix}
1 & 1\\
1 & -1
\end{bmatrix}\\
H_{2^k} &= \begin{bmatrix}
H_{2^{k-1}} & H_{2^{k-1}}\\
H_{2^{k-1}} & -H_{2^{k-1}}
\end{bmatrix}
\end{aligned}
\end{equation}
For the gene regulatory network module given by $\{l_1,l_2\}$, we can describe the transformation of coordinates from the probabilities $\vec{p}_{l_1 l_2} = (p_1, p_2, p_3, p_4)$ to the expectation values $\vec{E}_{l_1 l_2} = (1, \langle l_1 \rangle, \langle l_2 \rangle, \langle l_1 l_2 \rangle)$ using the Hadamard matrix $H_{2k}$ for $k=2$ as
\begin{equation}
\begin{aligned}\label{eq:expecttrans}
H_4 \vec{p}_{l_1 l_2} = \vec{E}.
\end{aligned}
\end{equation}
Since the inverse of the Hadamard transform is proportional to itself (i.e. $H_{2k}^{-1} = \frac{1}{2k}H_{2k}$) we can solve this system of equations to express each probability parameter in terms of expected value coordinates
\begin{equation}
\begin{aligned}\label{eq:expecttransfull}
\vec{p}_{l_1 l_2} &= H_4^{-1}\vec{E}\\
\begin{bmatrix}
p_1\\
p_2\\
p_3\\
p_4
\end{bmatrix} &= \frac{1}{4}\begin{bmatrix}
  1 & 1 & 1 & 1\\
  1 & -1 & 1 & -1\\
  1 & 1 & -1 & -1\\
  1 & -1 & -1 & 1\\
\end{bmatrix} \begin{bmatrix}
1\\
\langle l_1 \rangle\\
\langle l_2 \rangle\\
\langle l_1 l_2 \rangle
\end{bmatrix}
\end{aligned}
\end{equation}
Recall that these equations enforce normalization and Kolmogorov consistency. We can additionally enforce positivity, $p_i \geq 0$, in terms of the expectation value coordinates as
\begin{equation}
\begin{aligned}\label{eq:kcineq}
\begin{bmatrix}
  1 & 1 & 1 & 1\\
  1 & -1 & 1 & -1\\
  1 & 1 & -1 & -1\\
  1 & -1 & -1 & 1\\
\end{bmatrix} \begin{bmatrix}
1\\
\langle l_1 \rangle\\
\langle l_2 \rangle\\
\langle l_1 l_2 \rangle
\end{bmatrix} &\geq \begin{bmatrix}
0\\
0\\
0\\
0
\end{bmatrix}\\
H_4^{-1}\vec{E} &\geq \vec{0}
\end{aligned}
\end{equation}
Analogous equations hold for all other gene regulatory network modules given by $\{l'_1,l_2\}, \{l_1,l'_2\}, \{l'_1,l'_2\}$ yielding a total of $4n$ inequalities. Of these, only $2n$ are independent. Vertices of the polytopes corresponding to both the non-modular and modular spaces of probability distributions in terms of expectation value coordinates are of the form
\begin{equation}
\begin{aligned}\label{eq:expvalvert}
\left( \langle l_1 \rangle, \langle l_2 \rangle, \langle l'_1 \rangle, \langle l'_2 \rangle, \langle l_1 l_2 \rangle, \langle l'_1 l_2 \rangle, \langle l'_1 l'_2 \rangle, \langle l_1 l'_2 \rangle  \right).
\end{aligned}
\end{equation}
In the case at hand it turns out that there are $2^n + 2^{n-1}$ vertices of the polytope corresponding to the space of modular distributions and $2^n$ are strictly non-modular while $2^{n-1}$ are strictly modular. For the $(2,2,2)$ case at hand corresponding to the $4-$cycle graph this gives $24$ vertices for the modular polytope and $16$ vertices for the non-modular polytope. The $2^n$ non-modular vertices are given by
\begin{equation}
\begin{aligned}\label{eq:nonmodvert}
\left( \langle l_1 \rangle, \langle l_2 \rangle, \langle l'_1 \rangle, \langle l'_2 \rangle, \langle l_1 \rangle \langle l_2 \rangle, \langle l'_1 \rangle \langle l_2 \rangle, \langle l'_1 \rangle \langle l'_2 \rangle, \langle l_1 \rangle \langle l'_2 \rangle  \right).
\end{aligned}
\end{equation}
and the $2^{n-1}$ modular vertices are given by
\begin{equation}
\begin{aligned}\label{eq:modvert}
\left( 0, 0, 0, 0, \langle l_1 l_2 \rangle, \langle l'_1 l_2 \rangle, \langle l'_1 l'_2 \rangle, \langle l_1 l'_2 \rangle  \right)
\end{aligned}
\end{equation}
where the $\langle l_i l_{i+1} \rangle$ take values in the set $\{+1,-1\}$ such that there are an odd number of negatives in each of the vertices indicated in \ref{eq:modvert}. To be explicit, this results for the $(2,2,2)$ case of the $4-$cycle graph in the following $2^n + 2^{n-1}$ vertices for the modular polytope
\begin{equation}
\begin{aligned}\label{eq:modvertexp}
\begin{bmatrix}
  -1 & -1 & -1 & -1 & 1 & 1 & 1 & 1\\
  -1 & -1 & -1 & 1 & 1 & 1 & -1 & -1\\
  -1 & -1 & 1 & -1 & 1 & -1 & -1 & 1\\
  -1 & -1 & 1 & 1 & 1 & -1 & 1 & -1\\
  -1 & 1 & -1 & -1 & -1 & -1 & 1 & 1\\
  -1 & 1 & -1 & 1 & -1 & -1 & -1 & -1\\
  -1 & 1 & 1 & -1 & -1 & 1 & -1 & 1\\
  -1 & 1 & 1 & 1 & -1 & 1 & 1 & -1\\
  1 & -1 & -1 & -1 & -1 & 1 & 1 & -1\\
  1 & -1 & -1 & 1 & -1 & 1 & -1 & 1\\
  1 & -1 & 1 & -1 & -1 & -1 & -1 & -1\\
  1 & -1 & 1 & 1 & -1 & -1 & 1 & 1\\
  1 & 1 & -1 & -1 & 1 & -1 & 1 & -1\\
  1 & 1 & -1 & 1 & 1 & -1 & -1 & 1\\
  1 & 1 & 1 & -1 & 1 & 1 & -1 & -1\\
  1 & 1 & 1 & 1 & 1 & 1 & 1 & 1\\
  0 & 0 & 0 & 0 & -1 & 1 & 1 & 1\\
  0 & 0 & 0 & 0 & 1 & -1 & 1 & 1\\
  0 & 0 & 0 & 0 & 1 & 1 & -1 & 1\\
  0 & 0 & 0 & 0 & 1 & 1 & 1 & -1\\
  0 & 0 & 0 & 0 & -1 & -1 & -1 & 1\\
  0 & 0 & 0 & 0 & -1 & -1 & 1 & -1\\
  0 & 0 & 0 & 0 & -1 & 1 & -1 & -1\\
  0 & 0 & 0 & 0 & 1 & -1 & -1 & -1\\
\end{bmatrix}
\end{aligned}
\end{equation}
and $2^n$ vertices for the non-modular polytope
\begin{equation}
\begin{aligned}\label{eq:nonmodvertexp}
\begin{bmatrix}
  -1 & -1 & -1 & -1 & 1 & 1 & 1 & 1\\
  -1 & -1 & -1 & 1 & 1 & 1 & -1 & -1\\
  -1 & -1 & 1 & -1 & 1 & -1 & -1 & 1\\
  -1 & -1 & 1 & 1 & 1 & -1 & 1 & -1\\
  -1 & 1 & -1 & -1 & -1 & -1 & 1 & 1\\
  -1 & 1 & -1 & 1 & -1 & -1 & -1 & -1\\
  -1 & 1 & 1 & -1 & -1 & 1 & -1 & 1\\
  -1 & 1 & 1 & 1 & -1 & 1 & 1 & -1\\
  1 & -1 & -1 & -1 & -1 & 1 & 1 & -1\\
  1 & -1 & -1 & 1 & -1 & 1 & -1 & 1\\
  1 & -1 & 1 & -1 & -1 & -1 & -1 & -1\\
  1 & -1 & 1 & 1 & -1 & -1 & 1 & 1\\
  1 & 1 & -1 & -1 & 1 & -1 & 1 & -1\\
  1 & 1 & -1 & 1 & 1 & -1 & -1 & 1\\
  1 & 1 & 1 & -1 & 1 & 1 & -1 & -1\\
  1 & 1 & 1 & 1 & 1 & 1 & 1 & 1\\
\end{bmatrix}
\end{aligned}
\end{equation}
The $2^{n-1}$ vertices of the modular polytope excluded from the non-modular polytope violate the $2^{n-1}$ Boole inequalities determining some of the facets of the non-modular polytope. To be explicit, there are $2^n + 2^{n-1}$ facets of the non-modular polytope given by the following inequalities
\begin{equation}
\begin{aligned}\label{eq:nonmodineq}
\langle l_1 \rangle - \langle l_2 \rangle - \langle l_1 l_2 \rangle &\geq -1\\
\langle l_1 \rangle + \langle l_2 \rangle + \langle l_1 l_2 \rangle &\geq -1\\
\langle l_1 \rangle - \langle l'_2 \rangle - \langle l_1 l'_2 \rangle &\geq -1\\
\langle l_1 \rangle + \langle l'_2 \rangle + \langle l_1 l'_2 \rangle &\geq -1\\
-\langle l_1 \rangle + \langle l_2 \rangle - \langle l_1 l_2 \rangle &\geq -1\\
-\langle l_1 \rangle - \langle l_2 \rangle + \langle l_1 l_2 \rangle &\geq -1\\
-\langle l_1 \rangle + \langle l'_2 \rangle - \langle l_1 l'_2 \rangle &\geq -1\\
-\langle l_1 \rangle - \langle l'_2 \rangle + \langle l_1 l'_2 \rangle &\geq -1\\
\langle l_2 \rangle + \langle l'_1 \rangle + \langle l'_1 l_2 \rangle &\geq -1\\
\langle l_2 \rangle - \langle l'_1 \rangle - \langle l'_1 l_2 \rangle &\geq -1\\
-\langle l_2 \rangle + \langle l'_1 \rangle - \langle l'_1 l_2 \rangle &\geq -1\\
-\langle l_2 \rangle - \langle l'_1 \rangle + \langle l'_1 l_2 \rangle &\geq -1\\
\langle l'_1 \rangle - \langle l'_2 \rangle - \langle l'_1 l'_2 \rangle &\geq -1\\
\langle l'_1 \rangle + \langle l'_2 \rangle + \langle l'_1 l'_2 \rangle &\geq -1\\
-\langle l'_1 \rangle + \langle l'_2 \rangle - \langle l'_1 l'_2 \rangle &\geq -1\\
-\langle l'_1 \rangle - \langle l'_2 \rangle + \langle l'_1 l'_2 \rangle &\geq -1\\
\langle l_1 l_2 \rangle - \langle l'_1 l_2 \rangle + \langle l'_1 l'_2 \rangle + \langle l_1 l'_2 \rangle &\geq -2\\
\langle l_1 l_2 \rangle + \langle l'_1 l_2 \rangle - \langle l'_1 l'_2 \rangle + \langle l_1 l'_2 \rangle &\geq -2\\
\langle l_1 l_2 \rangle + \langle l'_1 l_2 \rangle + \langle l'_1 l'_2 \rangle - \langle l_1 l'_2 \rangle &\geq -2\\
\langle l_1 l_2 \rangle - \langle l'_1 l_2 \rangle - \langle l'_1 l'_2 \rangle - \langle l_1 l'_2 \rangle &\geq -2\\
-\langle l_1 l_2 \rangle + \langle l'_1 l_2 \rangle - \langle l'_1 l'_2 \rangle - \langle l_1 l'_2 \rangle &\geq -2\\
-\langle l_1 l_2 \rangle - \langle l'_1 l_2 \rangle + \langle l'_1 l'_2 \rangle - \langle l_1 l'_2 \rangle &\geq -2\\
-\langle l_1 l_2 \rangle + \langle l'_1 l_2 \rangle + \langle l'_1 l'_2 \rangle + \langle l_1 l'_2 \rangle &\geq -2\\
-\langle l_1 l_2 \rangle - \langle l'_1 l_2 \rangle - \langle l'_1 l'_2 \rangle + \langle l_1 l'_2 \rangle &\geq -2
\end{aligned}
\end{equation}
For the modular polytope, there are $2^n$ inequalities
\begin{equation}
\begin{aligned}\label{eq:modineq}
\langle l_1 \rangle + \langle l'_2 \rangle + \langle l_1 l'_2 \rangle &\geq -1\\
\langle l_1 \rangle + \langle l_2 \rangle + \langle l_1 l_2 \rangle &\geq -1\\
\langle l_1 \rangle - \langle l'_2 \rangle - \langle l_1 l'_2 \rangle &\geq -1\\
\langle l_1 \rangle - \langle l_2 \rangle - \langle l_1 l_2 \rangle &\geq -1\\
-\langle l_1 \rangle + \langle l_2 \rangle - \langle l_1 l_2 \rangle &\geq -1\\
-\langle l_1 \rangle - \langle l_2 \rangle + \langle l_1 l_2 \rangle &\geq -1\\
-\langle l_1 \rangle - \langle l'_2 \rangle + \langle l_1 l'_2 \rangle &\geq -1\\
-\langle l_1 \rangle + \langle l'_2 \rangle - \langle l_1 l'_2 \rangle &\geq -1\\
\langle l_2 \rangle + \langle l'_1 \rangle + \langle l'_1 l_2 \rangle &\geq -1\\
-\langle l_2 \rangle + \langle l'_1 \rangle - \langle l'_1 l_2 \rangle &\geq -1\\
-\langle l_2 \rangle - \langle l'_1 \rangle + \langle l'_1 l_2 \rangle &\geq -1\\
\langle l_2 \rangle - \langle l'_1 \rangle - \langle l'_1 l_2 \rangle &\geq -1\\
\langle l'_1 \rangle - \langle l'_2 \rangle - \langle l'_1 l'_2 \rangle &\geq -1\\
-\langle l'_1 \rangle + \langle l'_2 \rangle - \langle l'_1 l'_2 \rangle &\geq -1\\
\langle l'_1 \rangle + \langle l'_2 \rangle + \langle l'_1 l'_2 \rangle &\geq -1\\
-\langle l'_1 \rangle - \langle l'_2 \rangle + \langle l'_1 l'_2 \rangle &\geq -1
\end{aligned}
\end{equation}
The modular polytope is thus not required to respect the $2^{n-1}$ Boole inequalities which enables it to possess the $2^{n-1}$ additional vertices specified in \ref{eq:modvertexp} with respect to those specified in \ref{eq:nonmodvertexp}. The relationship between the vertex (V) and facet (H) representation given by the explicit enumeration of vertices \ref{eq:modvertexp} and \ref{eq:nonmodvertexp} and inequalities \ref{eq:modineq} and \ref{eq:nonmodineq} have been verified using polymake \cite{GawrilowEwgenijD-TJoswig2000}.

We can additionally use polymake to compute the volume of each polytope and thus evaluate the relative size of the space of models contained within each of the modular and non-modular polytopes. We can compute this volume and thus compare models for arbitrary $n-$cycle graphs. The volume ratios between the non-modular and modular polytopes were computed using polymake for $n=\{3,4,5,6\}$ and are displayed in Figure \ref{fig:ncycvolrat}.
