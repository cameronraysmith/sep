\begin{table}
\centering
\begin{tabular}{ r || c }
$l_1 l_2 l'_1 l'_2$ & probability \\ \hline
0000 & $q_1$ \\
0010 & $q_2$ \\
0001 & $q_3$ \\
0011 & $q_4$ \\
1000 & $q_5$ \\
1010 & $q_6$ \\
1001 & $q_7$ \\
1011 & $q_8$ \\
0100 & $q_9$ \\
0110 & $q_{10}$ \\
0101 & $q_{11}$ \\
0111 & $q_{12}$ \\
1100 & $q_{13}$ \\
1110 & $q_{14}$ \\
1101 & $q_{15}$ \\
1111 & $q_{16}$
\end{tabular}
\caption{Example of a distribution defined on the collection of maps from the set of all possible alleles, $L$, to the set of phenotype values, $P$. Such a distribution may be an instance of a global section $d \in \mathcal{D}_R\mathcal{E}(L)$ if it satisfies the sheaf condition.  Here each row represents the probability assigned to a map in the collection of maps given by the set $P^L$. For example, $P[l_1=0,l_2=0,l'_1=0,l'_2=0]=q_1$ gives the probability associated to the map $\{l_1, l_2, l'_1, l'_2\} \mapsto \{0,0,0,0\}$.}
\label{tab:hidvarmod}
\end{table}