%!TEX root = ../plos_template.tex

\section{Gene network-phenotype maps}\label{sec:genenetworkphenmap}
The genotype of an organism has a relatively straightforward definition in terms of the sequence of nucleotides comprising its genome. Phenotypes, on the other hand, can be described at different levels of organization~\cite{Dawkins1982,Stadler2001}. The concept of phenotype was initially defined at the level of macroscopically observable physical characteristics such as shape, size, color, and various combinations thereof~\cite{Johannsen1911}. However, since the advent of molecular biology, the lowest level description of phenotype might be considered to be the dynamic phenomenon that can be described by measuring the transcription states of all genes comprising an organism's genome.  These expression levels of subsets of interacting genes determine which enzymes are produced, thus determining the rate at which metabolic reactions proceed.  These reaction rates constitute the next level of phenotypes.  These in turn determine the higher level phenotypes, ultimately culminating in macroscopic phenotypes.  In order to discuss this notion of phenotypic levels in a more precise way, we develop a formal description of the \gnpm{} that takes into account higher-order correlations among genes as well as evidence that gene expression is either stochastic or is more effectively modeled as such.

\ref{fig:expression_concept}A shows a simplified representation of a plasmid encoding a small gene regulatory network (GRN) whose correlation strengths are not known but are to be derived from observation of the transcription process. The amount of a given transcript present in a cell given in terms of the discrete counts obtained via sequence census methods (e.g. RNA-seq) or relative abundance derived from microarray data can be binned into a smaller number of discrete classes by setting a collection of thresholds on the original data set. If only a single threshold is given, then the data can be binned into two classes depending upon whether or not the original measurement surpasses the given threshold in \ref{fig:expression_concept}B.
The time series that results from such observations can be used to infer various statistics that characterize gene expression such as correlations between pairs of genes. The statistics associated to any higher-level phenotype that is determined by a particular gene expression pattern should be functions of such statistics.

If a large enough number of thresholds is available to distinguish among all possible molecule counts, then this observational protocol becomes complementary to mechanistic models.  There may be several sources for stochasticity in \gnpm{} including small numbers of the causal molecules and products of gene expression as well as environmental fluctuations upon which \gnpm{} are conditioned~\cite{Swain2002,Paulsson2004,Thattai2004,Acar2008a,Lestas2010,Munsky2012,Chalancon2012,Neuert2013,Sanchez2013}. Regardless of the fundamental nature of stochasticity to gene expression, empirically, we observe statistical as opposed to deterministic data, and thus we focus here on stochastic models. For our purposes we assume that we are dealing with a stationary stochastic process. Mathematically, such a model may take the form of a Markov chain whose dynamics are governed by a master equation for probability distributions over molecule counts. For example, in the case of a three gene network, the master equation takes the form
$$
\frac{dP(n_1,n_2,n_3)}{dt} = \sum_{n'_1}\sum_{n'_2}\sum_{n'_3} M^{n_1\,n_2\,n_3}_{n'_1\,n'_2\,n'_3}(k) P(n'_1,n'_2,n'_3)
$$
where $P(n_1,n_2,n_3)$ gives the probability of observing $n_1$, $n_2$, and $n_3$ molecules of each of the three genes respectively and $M(k)$ is a Markov transition rate matrix that depends upon some rate functions $k$ that are determined by the network architecture and the dynamics of the interactions.  The solution to this equation will converge towards  a stationary distribution $P_s$ in the limit of long times. We can then coarse-grain over the molecule counts using a function that maps the states represented by vectors of integers into some other variables. For example, if $n_i$ are positive integers, denoting the latter $\mathbb{Z}^+$, then a function $f \colon \mathbb{Z}^+ \rightarrow \{0,1\}$ that takes any integer less than or equal to some threshold $T \in \mathbb{Z}^+$ to $0$ and any integer greater than $T$ to $1$ is a very simple example of such a coarse-graining. For this specific form of the coarse-graining function $f$, the coarse-grained stationary probability distribution takes the form
$$
P_s(b_1,b_2,b_3) = \sum_{n_1 \in f^{-1}(b_1)}\sum_{n_2 \in f^{-1}(b_2)}\sum_{n_3 \in f^{-1}(b_3)} P_s(n_1,n_2,n_3),
$$
where $b_1,b_2,b_3 \in \{ 0,1 \}$. It is also possible to consider the case where each gene is coarse-grained according to a different threshold and into a different number of classes.

At the lowest-level of the \gnpm{}, these coarse-grained expression values can be viewed as collectively determining the lowest level in the aforementioned hierarchy of phenotypes. In this way, the data derived from a number of samples of a given network can be described using binary sequences as in \ref{fig:expression_concept}C.  Summarizing this, we will assume that we have a finite set $L$ of genes and a finite set $P$ of coarse-grained expression levels. These expression levels can also be viewed as complementary to promoter states, and, realistically, the number of these is likely to be larger than two \cite{Rieckh2013a}.  Then a possible expression state of our genome is represented by a function $e : L \to P$ and coarse-graining a stationary distribution will lead to a probability distribution on the set of all maps, denoted $P^L$, from genes to expression states.

\section{Probability distributions over network modules}\label{sec:probabilitydistributionsonnetworks}
Here we describe examples of probability distributions over network modules. A general presentation is provided in \refsupp{} \ref{secsupp:probabilitydistributionsonnetworks}.

In addition to the expression state of the whole genome, we usually will be interested in the expression states of portions of the genome that interact either directly or indirectly.  We will represent these portions as subsets of $L$ and their expression states as functions from these subsets to $P$. If $O \subset L$, let $\expr{}(O)$ denote the set of all possible functions from $O$ to $P$.

% The power set of $L$, which we shall denote as $\mathcal{P}(L)$, can be regarded  as a category~\cite{Lane1998,MacLane1992,Awodey2006,Abramsky2011} in which the objects are subsets of $L$ and morphisms represent inclusion of a smaller subset into a larger superset (i.e. $O \subseteq O' \Rightarrow O \rightarrow O'$).
If we consider the case in which we have two genes $L=\{l_1,l_2\}$ and there are two potential phenotypes, $P=\{0,1\}$, then $\expr$ operates on each of the possible subsets of $L$ (i.e. $\{\}$, $\{l_1\}$, $\{l_2\}$, $\{l_1,l_2\}$) to give spaces of functions containing the possible \gnpm{} as exemplified in \ref{fig:efunctor}. For example, $\mathcal{E}(\{l_1,l_2\}) = \{ e^{12}_{00},e^{12}_{01},e^{12}_{10},e^{12}_{11} \}$ where $e^{12}_{01}(l_1) = 0$ and $e^{12}_{01}(l_2) = 1$. As another example, $\mathcal{E}(\{l_1\}) = \{ e^{1}_{0},e^{1}_{1} \}$ where $e^{1}_{0}(l_1)=0$ and $e^{1}_{1}(l_1)=1$.

Given a finite set $S$, define $\dist(S)$ to be the set of all probability distributions on $S$. Continuing the example above,
$$\mathcal{D}(\mathcal{E}(\{l_1,l_2\})) = \{p^{12}_{00},p^{12}_{01},p^{12}_{10},p^{12}_{11} \mid p^{12}_{00} \geq 0, p^{12}_{01} \geq 0,p^{12}_{10} \geq 0,p^{12}_{11} \geq 0, p^{12}_{00} + p^{12}_{01} + p^{12}_{10} + p^{12}_{11} = 1 \},$$
$$\mathcal{D}(\mathcal{E}(\{l_1\})) = \{p^{1}_{0}, p^{1}_{1} \mid p^{1}_{0} \geq 0, p^{1}_{1} \geq 0, p^{1}_{0}+p^{1}_{1} = 1 \}.$$

If $S$ and $S'$ are finite sets, which in our case will usually be sets of \gnpm{} given by $\mathcal{E}(O)$, then $d \in \dist(S)$ and $d' \in \dist(S')$ are probability distributions. If $f \colon S \to S'$
is an onto map, then $\dist{}(f)$ is defined as marginalization. For example, in case $S=\{ e^{12}_{00}, e^{12}_{01}, e^{12}_{10}, e^{12}_{11} \}$, $S'=\{ e^{1}_{0}, e^{1}_{1} \}$, and $f$ is given by
\begin{equation}
\begin{aligned}
e^{12}_{00} \mapsto e^{1}_{0}\\
e^{12}_{01} \mapsto e^{1}_{0}\\
e^{12}_{10} \mapsto e^{1}_{1}\\
e^{12}_{11} \mapsto e^{1}_{1}
\end{aligned},
\end{equation}
then $\dist{}(f)$ can be represented as a marginalization matrix
\begin{equation}
\begin{pmatrix}
p^{1}_{0}\\
p^{1}_{1}
\end{pmatrix} = \begin{pmatrix}
1 & 1 & 0 & 0\\
0 & 0 & 1 & 1
\end{pmatrix}
\begin{pmatrix}
p^{12}_{00}\\
p^{12}_{01}\\
p^{12}_{10}\\
p^{12}_{11}
\end{pmatrix}.
\end{equation}

\section{Coarse-graining phenotypes}\label{sec:coarsegrainingphenotypes}

As described in \ref{sec:genenetworkphenmap} it is also possible to consider phenotypes that derive from coarse-graining lower-level phenotypes. Once this is done, one arrives at probability distributions over network modules like that introduced in \ref{sec:probabilitydistributionsonnetworks}. As a result of this, our conclusions that are formulated in terms of a single level of coarse-graining \gnpm{} also apply to coarse-graining over multiple levels at once despite the fact that the parameters of the relevant probabilistic model are likely to be different.

For a subset of genes $O \subseteq L$, let $\phi_i (O)$ be the set of phenotype values at level $i$, which can be determined from the expression levels of genes in $O$.  Note that $\phi_i (O)$ may be empty if the set $O$ does not contain enough genes to determine the values of any phenotype at level $i$. If $i \le j$, let $\Omega_{ij}(O) : \phi_i(O) \to \phi_j(O)$ be the coarse-graining map which describes how higher level phenotypes are determined from lower level phenotypes. This map is formulated precisely in \refsupp{} \ref{secsupp:coarsegrainingphenotypes}.

For example, if our lower level phenotypes for a set of genes $O_1 = \{ l_1,l_2,l_3,l_4 \}$ are given by a set of binary sequences, then the projection of these phenotypes down to the set $O_2 = \{l_3,l_4\}$ followed by mapping to the higher level phenotypes $x=\{01,10\}$ and $y=\{11\}$ is equivalent to first mapping to the higher-level phenotypes $X$ and $Y$ and then projecting down to $O_2$ shown by the equivalent paths from the top-left to the bottom-right in \ref{fig:phenotypehierarchy}A.
Of course, there is an equivalent diagram for the subset $\{ l_1,l_2,l_3 \}$.

\section{Gene regulatory network modules}\label{sec:covergenotypespace}
A gene regulatory network module is represented by a subset of genes, $O \subseteq L$. A gene regulatory network architecture (GRNA) may then be represented by a subset of such modules, $\mathcal{G} \subseteq \mathcal{P}(L)$ that satisfies the following two conditions
\begin{enumerate}
\item $\cup_i O_i = \cup \mathcal{G} = L$,
\item If $O,O' \in \mathcal{G}$ and $O \subseteq O'$ then $O = O'$.
\end{enumerate}
The first condition is just a statement that $\mathcal{G}$ represents a decomposition of the collection of all genes under consideration into subsets and this is why we refer to $\mathcal{G}$ as a collection of gene regulatory network modules. The second condition means simply that we will not consider nested subsets and so we will take for our $O \in \mathcal{G}$ the biggest $O \in \mathcal{G}$ that is not a subset of some other $O' \in \mathcal{G}$. The second condition also implies that if a given subset of genes $O'$ is compatible in a sense to be explained more precisely in what proceeds then any smaller subset of genes $O$ is also compatible.

Mathematically, the two conditions given above state that $\mathcal{G}$ is a \emph{covering} of the set $L$.  This is equivalent to $(L, \mathcal{G})$ being a reduced hypergraph, Sperner family, or clutter over $L$ \cite{Lauritzen1996}.  Coverings $\mathcal{G}$ of the space of genes contain the necessary information to make precise what we heuristically refer to at other points in this paper as modularity in order to cohere with standard terminology in systems biology literature while attempting to submit our own precise interpretation of the relatively colloquial concept.

Given a covering $\mathcal{G}$ of the space of genes, we can consider the higher order phenotypes associated to the elements of $\mathcal{G}$.  For a suitable choice of cover and a suitable level of phenotypes, it may happen that the phenotypes associated to different elements of $\mathcal{G}$ are distinct.  For instance, in the example of \ref{fig:phenotypehierarchy}, if we take $\mathcal{G} = \{O_1, O_2\}$ where $O_1 = \{l_1, l_2, l_3\}$ and $O_2 = \{l_3, l_4\}$, we have $\phi_{i+1}(O_1) = \{u,v\}$ and $\phi_{i+1}(O_2) = \{x,y\}$.  In such a case, if we were to perform one experiment which measured the phenotypes $\{u,v\}$ and another experiment which measured $\{x,y\}$, then the result could be understood as examining the covering $\{O_1, O_2\}$ at phenotype level $i+1$.

\section{Compatibility of distributions on \gnpm{}}\label{sec:compatibilityofgpms}
Given a covering of the space of genes $\mathcal{G}$, a compatible family for $\mathcal{G}$ with respect to $\mathcal{D} \circ \mathcal{E}$ is given by a family of distributions $\dist{}(\expr{}(\mathcal{G})) = \{d_O \in \dist (\mathcal{E}(O)) | O \in \mathcal{G}\}$ such that for all $O, O' \in \mathcal{G}$
\begin{eqnarray}\label{eq:sheafcond}
%\forall O \in \mathcal{G} \left[ \exists d_O \in \mathcal{D}_R\mathcal{E}(O) \right],\\
d_O|O \cap O' = d_{O'}|O \cap O'.
\end{eqnarray}
This first set of conditions is later referred to as local consistency. The space of all such locally consistent distributions for a given covering, $\mathcal{G}$, is referred to as $\mathbb{L}(\mathcal{G})$ where
\begin{equation}
\mathbb{L}(\mathcal{G}) = \{ d_O  \in \dist{}(\expr{}(\mathcal{G})) \mid (\forall O,O' \in \mathcal{G})\,\, d_O|O \cap O' = d_O'|O \cap O' \}.
\end{equation}
These conditions mean that any two distributions $d_O$ and $d_{O'}$ in the \emph{compatible family} of distributions marginalize to the same disribution over the intersection of $O$ with $O'$. If these constraints are not satisfied, then there is no way to make a consistent assignment of probabilities to the expression states of even a single gene. In this case one of the constraints will be eliminated or gene duplication may allow for the independent satisfaction of both constraints.

If, moreover, this first condition implies the existence of $d \in \mathcal{D}( \mathcal{E}(L))$ such that $d|O = d_O$ for all $O \in \mathcal{G}$ then the system is said to satisfy the global consistency condition. The space of all such globally consistent distributions for a given covering, $\mathcal{G}$, is referred to as $\mathbb{M}(\mathcal{G})$ where
\begin{equation}
\mathbb{M}(\mathcal{G}) = \{ d_O \in \dist{}(\expr{}(\mathcal{G})) \mid (\exists d ) \,\, d|O = d_O \}.
\end{equation}
In general, the system of equations $d|O = d_O$ for all $O \in \mathcal{G}$ is underdetermined and so local consistency does not imply global consistency. The local and global consistency conditions are related to the mathematical concepts from which they derive in \refsupp{} \ref{secsupp:compatibilityofgpms}. We will now examine the situation more closely to determine under what circumstances global consistency holds.

For a cover of the genotype space, $\mathcal{G}$, we can construct a matrix, $\mathbf{G}$, representing the relationship,
% $R = \coprod_{O \in \mathcal{G}} \mathcal{E}(O \subset L) \subseteq \mathcal{E}(L) \times \expr (\mathcal{G})$,
between \gnpm{} having as domain particular gene regulatory network modules given by the $O \in \mathcal{G}$ and those global \gnpm{} defined on $L$.
% We would like to construct the matrix representation of $\mathbf{G}$. In the first factor, $\mathcal{E}(L) = P^L = \{e^{L}_{\vec{j}} | \vec{j} \in P^{|L|}\}$.
%%$\{t_1, \ldots t_m\} = \{t_j | t_j \in P^L\}$.
% For the second factor, $ \mathcal{E}(\mathcal{G}) = \coprod_{O \in \mathcal{G}} \mathcal{E}(O) = \{e^O_{\vec{i}} | O \in \mathcal{G}, \vec{i} \in P^{|O|} \}$.
% %$\{e_1, \ldots e_n\} = \coprod_{O \in \mathcal{G}} \mathcal{E}(O)$.
%  So we have two sets of maps, one defined on $P^L$ and the other defined on $\mathcal{E}(O) = P^O$ for each $O \in \mathcal{G}$. This yields a method of specifying the intended relationship that defines
$\mathbf{G}$ can be specified for all $e^O_{\vec{i}} \in \mathcal{E}(\mathcal{G})$ and $e^{L}_{\vec{j}} \in \mathcal{E}(L)$:
\begin{eqnarray}\label{eq:margmat}
\mathbf{G}(e^O_{\vec{i}},e^L_{\vec{j}}) =
\begin{cases}
1, & e^L_{\vec{j}}|O = e^O_{\vec{i}},\\
0, & \text{otherwise}.
\end{cases}
\end{eqnarray}
For example, given the covering $\mathcal{G} = \{ \{l_1\}, \{l_2\} \}$ of a set of two genes $L = \{ l_1, l_2 \}$ the associated matrix $\mathbf{G}$ is shown in \ref{fig:efunctor}B.

% Combining the marginalization matrix $\mathbf{G}$ with the maps previously defined in \ref{eq:emb} the following diagram demonstrates the relationships between the spaces of probability distributions and the linear spaces in which they are embedded

% \begin{equation*}
% \xymatrix{
%  \mathbb{R}^{\mathcal{E}(L)} \ar[r]^{\mathbf{G}} &
%    \mathbb{R}^{\mathcal{E}(\mathcal{G})} \\
%  \dist (\mathcal{E}(L)) \ar[r]^{\mathbf{G}} \ar@{^{(}->}[u]^{emb_{\mathcal{E}(L)}}& \ar@{^{(}->}[u]_{emb_{\mathcal{E}(\mathcal{G})}}
%   \dist (\mathcal{E}(\mathcal{G}))
%   }
% \end{equation*}
The locally, $\mathbb{L}(\mathcal{G})$, and globally, $\mathbb{M}(\mathcal{G})$, consistent polytopes correspond to the spaces of probability distributions satisfying the local and global consistency conditions described above. The globally consistent polytope is a subspace of the locally consistent one. This is because an ostensible inverse to $\mathbf{G}$ is underdetermined and can only be estimated in general (most commonly using the maximum entropy principle).
% In terms of the diagram $\mathbb{M}(\mathcal{G})$ is given by $\mathbf{G}(emb_{\mathcal{E}(L)}(\mathcal{D}(\mathcal{E}(L))))$ and $\mathbb{L}(\mathcal{G})$ is given by $emb_{\mathcal{E}(\mathcal{G})}(\mathcal{D}(\mathcal{E}(\mathcal{G}))) \cap \mathbf{G}(\mathbb{R}^{\mathcal{E}(L)})$, where
% \begin{eqnarray}
% emb_{\mathcal{E}(\mathcal{G})}(\mathcal{D}(\mathcal{E}(\mathcal{G}))) &=& \left\{ p^{O}_{\vec{i}} \; \bigg| \; (\forall\, O \in \mathcal{G}) \; (\forall\, \vec{i} \in P^{|O|}) \;\; p^O_{\vec{i}} \geq 0, \; (\forall\, O \in \mathcal{G}) \sum_{\vec{i} \in P^{|O|}} p^{O}_{\vec{i}} = 1 \right\},\\
% emb_{\mathcal{E}(L)}(\mathcal{D}(\mathcal{E}(L))) &=& \left\{ p^L_{\vec{i}} \; \bigg| \; (\forall\, \vec{i} \in P^{|L|}) \; p^L_{\vec{i}} \geq 0, \; \sum_{\vec{i} \in P^{|L|}} p^L_{\vec{i}} = 1 \right\}.
% \end{eqnarray}
To determine explicit conditions on the probabilities we express $\mathbb{L}(\mathcal{G})$ and $\mathbb{M}(\mathcal{G})$ in terms of the fundamental subspaces associated to the linear map $\mathbf{G}$. In order for a vector $v$ to lie in $\mathbf{G}(\mathbb{R}^{\mathcal{E}(L)})$, we must have $v=\mathbf{G}x$ for some $x \in \mathbb{R}^{\mathcal{E}(L)}$.  The cokernel of $\mathbf{G}$ gives the obstructions to this system $v=\mathbf{G}x$ having a solution. In order to eliminate these obstructions, constraints must be imposed on $\mathbb{R}^{\mathcal{E}(\mathcal{G})}$ and these constraints are given precisely via annihilating the cokernel, i.e. $\mathbf{G}(\mathbb{R}^{\mathcal{E}(L)}) = \{ v \; | \; (\forall u \in coker \mathbf{G}) \; u \cdot v = 0\}$.  We then take the appropriate intersection to determine $\mathbb{L}(\mathcal{G})$ by requiring $v \in \mathcal{D}(\mathcal{E}(\mathcal{G}))$
\begin{eqnarray}\label{eq:localpolytope}
\mathbb{L}(\mathcal{G}) &=& \{ v \in \dist{}(\expr{}(\mathcal{G})) \; | \; (\forall u \in coker \mathbf{G}) \; u \cdot v = 0\}.
\end{eqnarray}
Since $\mathbf{G}$ is not invertible the equation $v = \mathbf{G}x$ can only be solved up to an element of $ker \mathbf{G}$. $v = \mathbf{G}x$ can thus be solved on a subspace $T$ of $\mathbb{R}^{\mathcal{E}(L)}$ such that $T \oplus ker \mathbf{G} = \mathbb{R}^{\mathcal{E}(L)}$ to yield
\begin{eqnarray}\label{eq:globalpolytope}
\mathbb{M}(\mathcal{G}) &=& \{ v \; | v = \mathbf{G} x,\; (\exists x \in T) \; (\exists y \in \dist{}(\expr{}(L)))\; x-y \in ker \mathbf{G}  \}.
\end{eqnarray}
In order to obtain inequalities that define $\mathbb{M}(\mathcal{G})$, Fourier-Motzkin elimination can be used to eliminate $x$ and $y$. Alternatively one can use the fact, \cite{Wainwright2007} proposition 8.3, that $\mathbb{M}(\mathcal{G})$ is given by removing the non-integer vertices from a vertex represntation of $\mathbb{L}(\mathcal{G})$ and the ability to interconvert between vertex and inequality representations to compute the same inequalities as described in \refsupp{} \ref{secsupp:fourcycleexample}.

\subsection{Example of unsatisfiable constraints}\label{sec:inconsistency}
We will now exemplify equations and inequalities that need to be satisfied in order to guarantee the sheaf condition for the simple case of three genes that form the simplest nontrivial cycle where inconsistency may arise. Suppose that $L = \{l_1,l_2,l_3\}$, $P = \{0,1\}$, $\mathcal{G} = \{\{l_1,l_2\},\{l_2,l_3\},\{l_3,l_1\}\}$. We define probabilities associated to the gene network-phenotype maps for any set $S$ as $p^S_{\vec{j}}= d_{S}(e^S_{\vec{j}})$. This results in $\mathcal{D}(\mathcal{E}(L)) = \Delta_7 \subset \mathbb{R}^8 $ and $\mathcal{D}(\mathcal{E}(\mathcal{G})) = \Delta^{\oplus 3}_3 \subset \mathbb{R}^{12}$,
which give the inequalities defining $\mathbb{L}(\mathcal{G})$ and $\mathbb{M}(\mathcal{G})$ when substituted into \ref{eq:localpolytope} and \ref{eq:globalpolytope} with the appropriate $\mathbf{G}$ given by \ref{eq:margmat}.

An example of normalized contingency tables that correspond to this model are presented in \ref{fig:inconsistentthreecycle}A.
One local consistency condition in terms of \ref{eq:sheafcond} specifies that for $e_{\{l_1\}} \in \mathcal{E}(\{l_1\})$, which assigns a particular phenotype to the genotype $\{l_1\}$ rather than to an entire gene regulatory network module like $O$ or $O'$, that
\begin{eqnarray}\label{eq:sheafprob}
\dist (\expr( \{ l_1 \} \subset O ) )(d_O)(e_{\{l_1\}}) = \sum_{\substack{ e \in \mathcal{E}(O), \\  e|l_1=e_{\{l_1\}}} } d_O(e) \,\, = \sum_{\substack{ e' \in \mathcal{E}(O'), \\ e'|l_1=e_{\{l_1\}}} } d_{O'}(e') = \dist (\expr( \{ l_1 \} \subset O' ) )(d_{O'})(e_{\{l_1\}})
\end{eqnarray}
This condition means that the probability for gene $l_1$ to be associated to the phenotype given by $e_{\{l_1\}}$ is equivalent in case we marginalize over all the other genes contained in the gene regulatory network modules of which $l_1$ is a component.  \ref{eq:sheafprob} reduces to two equations corresponding to the case when $e_{l_1}$ is the map which sends $l_1$ to $0$ and the case when $e_{l_1}$ sends $l_1$ to $1$. If we do likewise with $l_2$ and $l_3$ in place of $l_1$ we obtain the set of local consistency conditions:
\begin{equation}
\begin{aligned}\label{eq:localconsistencythreegenes}
 p^{12}_{00} + p^{12}_{01} &= p^{1}_0 = p^{13}_{00} + p^{13}_{01}, &
 p^{12}_{00} + p^{12}_{10} &= p^{2}_0 = p^{23}_{00} + p^{23}_{01}, &
 p^{13}_{00} + p^{13}_{10} &= p^{3}_0 = p^{23}_{00} + p^{23}_{10},\\
 p^{12}_{10} + p^{12}_{11} &= p^{1}_1 = p^{13}_{10} + p^{13}_{11}, &
 p^{12}_{01} + p^{12}_{11} &= p^{2}_1 = p^{23}_{10} + p^{23}_{11}, &
 p^{13}_{01} + p^{13}_{11} &= p^{3}_1 = p^{23}_{01} + p^{23}_{11}.
 \end{aligned}
 \end{equation}
These are equivalent to what would result from using the cokernel of $\mathbf{G}$ in \ref{eq:localpolytope}.
Using the local consistency conditions for our example we can derive a set of inequalities that determine $\mathbb{L}(\mathcal{G})$
\begin{equation}
\begin{aligned}\label{eq:threecycinequalities}
p^{12}_{00} &= 1 + p^{12}_{11} - p^{23}_{10} - p^{23}_{11} - p^{13}_{10} - p^{13}_{11} \geq 0, \\
p^{12}_{01} &= -p^{12}_{11} + p^{23}_{10} + p^{23}_{11} \geq 0,\\
p^{12}_{10} &= -p^{12}_{11} + p^{13}_{10} + p^{13}_{11} \geq 0,\\
p^{23}_{00} &= 1-p^{23}_{10} - p^{13}_{01} - p^{13}_{11} \geq 0,\\
p^{23}_{01} &= -p^{23}_{11} + p^{13}_{01} + p^{13}_{11} \geq 0,\\
p^{31}_{00} &= 1-p^{13}_{10} - p^{13}_{01} - p^{13}_{11} \geq 0,
\end{aligned}
\end{equation}
combined with the trivial inequalities that force all probabilities to be nonnegative. Substituting the numbers from \ref{fig:inconsistentthreecycle}A into \ref{eq:threecycinequalities}, demonstrates that the local conditions are satisfied.

The global consistency conditions form an underdetermined system of linear equations for the putative global distribution $d_{L}$ so their solution will assume the form of a linear subspace as demonstrated in \ref{eq:globalpolytope}.  In our example where this linear subspace corresponds to the image of $\Delta_7$ in $\Delta_3^{\oplus 3}$ under the marginalization map, following \ref{eq:globalpolytope} we can choose $v=p^0_{\vec{i}}$, $y=p^L_{\vec{i}}$, and $T$ by fixing $p^{123}_{000}=0$ we get the following by eliminating $x$ from the equations determined by the conditions $v=Gx$, $x \in T$, $x-y \in ker \mathbf{G}$:
\begin{equation}
\begin{aligned}\label{eq:globalpositivityeqs}
p^{123}_{001} &= p^{12}_{00} - p^{123}_{000} \\
p^{123}_{010} &= p^{13}_{00} - p^{123}_{000} \\
p^{123}_{100} &= p^{23}_{00} - p^{123}_{000} \\
p^{123}_{110} &= p^{23}_{10} - p^{13}_{00} + p^{123}_{000} \\
p^{123}_{011} &= p^{13}_{01} - p^{12}_{00} + p^{123}_{000} \\
p^{123}_{101} &= p^{12}_{10} - p^{23}_{00} + p^{123}_{000} \\
p^{123}_{111} &= 1 - p^{12}_{00} - p^{13}_{00} - p^{23}_{00} - p^{123}_{000}
\end{aligned}
\end{equation}
The remaining condition $y \in \mathcal{D}(\mathcal{E}(L))$ from \ref{eq:globalpolytope} states that all the probabilities $p^{123}_{ijk}$ must be positive numbers, which is only possible if the putative marginals satisfy suitable inequalities given by
\begin{equation}
\begin{aligned}\label{eq:globalpositivityineqs}
p^{123}_{000} &\geq min(p^{12}_{00},\, p^{13}_{00},p^{23}_{00},\, 1 - p^{12}_{00} - p^{13}_{00} - p^{23}_{00}),\\
 p^{123}_{000} &\geq max(0,\, p^{13}_{00}-p^{23}_{10},\, p^{12}_{00}-p^{13}_{01},\, p^{23}_{00}-p^{12}_{10}).
\end{aligned}
\end{equation}
A minimal set of inequalities is then expressed by substituting the equalities from \ref{eq:threecycinequalities} into the inequalities determined by \ref{eq:globalpositivityineqs} and eliminating redundancies resulting in
\begin{equation}
\begin{aligned}\label{eq:threecycbooleinequalities}
p^{12}_{11} - p^{23}_{11} + p^{13}_{01} \geq 0, \\
1 + p^{12}_{11} - p^{23}_{10} - p^{13}_{10} - p^{13}_{01} - p^{13}_{11} \geq 0, \\
-p^{12}_{11} + p^{23}_{10} + p^{13}_{11} \geq 0, \\
-p^{12}_{11} + p^{23}_{11} + p^{13}_{10} \geq 0.
\end{aligned}
\end{equation}
The inequalities from \ref{eq:threecycinequalities} and \ref{eq:threecycbooleinequalities} combined with the nonnegativity inequalities together determine the global polytope $\mathbb{M}(\mathcal{G})$. For the example given in \ref{fig:inconsistentthreecycle}A, the first of the inequalities in \ref{eq:threecycbooleinequalities} is demonstrated to be unsatisfied in \ref{eq:threecycboolenumbers}
\begin{equation}
\begin{aligned}\label{eq:threecycboolenumbers}
0.1 - 0.4 + 0.1 \not\geq 0, \\
1 + 0.1 - 0.1 - 0.1 - 0.1 - 0.4 \geq 0, \\
-0.1 + 0.1 + 0.4 \geq 0, \\
-0.1 + 0.4 + 0.1 \geq 0.
\end{aligned}
\end{equation}
This indicates that data consistent with \ref{fig:inconsistentthreecycle}A could not derive from the network depicted there.
\subsection{Example of apparent satisfaction of unsatisfiable constraints}\label{sec:apparentinconsistency}
The inequalities defining $\mathbb{M}(\mathcal{G})$ were derived under the assumption that the two-element probabilities were obtained by mariginalizing a three-element distribution.  If some other procedure, such as conditionalization, is used to obtain them instead, these inequalities need not apply. For example, suppose now that $L = \{l_1,l_2,l_3 \}$, $P = \{0,1,2\}$, $\mathcal{G} = \{\{l_1,l_2\},\{l_2,l_3\},\{l_3,l_1\}\}$ where we have simply added an element to $P$ relative to the example described above. In the previous example the marginal maps were given by $\dist (\expr (O \subset L))$ with one for each $O \in \mathcal{G}$. If we combine these marginal maps with conditioning on one out of the three genes being in state two and each of the other two being in states zero or one, then we have instead $\dist (\pi_1), \dist (\pi_2), \dist (\pi_3)$ where $\pi_1 = \expr (\{l_1,l_2\} \subset L)|\{ e^{123}_{ij2} \mid i,j \in \{ 0,1 \} \},\; \pi_2 = \expr (\{l_2,l_3\} \subset L)|\{ e^{123}_{2ij} \mid i,j \in \{ 0,1 \} \},\; \pi_3 = \expr (\{l_3,l_1\} \subset L)|\{ e^{123}_{i2j} \mid i,j \in \{ 0,1 \} \}$. In this case, if we have the following assignment of probabilities for a distribution $d$
\begin{equation}\label{eq:condprobs}
\begin{aligned}
p^{123}_{002} &= 1/30 & p^{123}_{020} &= 2/15 & p^{123}_{200} &= 2/15\\
p^{123}_{012} &= 2/15 & p^{123}_{021} &= 1/30 & p^{123}_{201} &= 1/30\\
p^{123}_{102} &= 2/15 & p^{123}_{120} &= 1/30 & p^{123}_{210} &= 1/30\\
p^{123}_{112} &= 1/30 & p^{123}_{121} &= 2/15 & p^{123}_{211} &= 2/15
\end{aligned}
\end{equation}
with all other probabilities being zero, then $\dist (\pi_1)(d), \dist (\pi_2)(d), \dist (\pi_3)(d)$ are equivalent to the probability tables in \ref{fig:inconsistentthreecycle}A, which as shown in \ref{sec:inconsistency}, could not be achieved by marginalization alone. For example, given that $dom(\pi_1) = \{ e^{123}_{002}, e^{123}_{012}, e^{123}_{102}, e^{123}_{112} \}$ then $d(dom(\pi_1)) = \frac{1}{30} + \frac{2}{30} + \frac{2}{30} + \frac{1}{30} = \frac{1}{3}$. Substituting this factor and the fact that ${\pi_1}^{-1}(e^{12}_{ij}) = e^{123}_{ij2}$ into \ref{eq:distfunctor}
% $$
% p^{12}_{ij} = \dist (\pi_1)(d)(e^{12}_{ij}) = \frac{d({\pi_1}^{-1}(e^{12}_{ij}))}{d(dom(\pi_1))} = 3 p^{123}_{ij2},
% $$
$$
p^{12}_{ij} = p(l_1 = i , l_2 = j \mid l_3 = 2) = \frac{p^{123}_{ij2}}{p^{123}_{002}+p^{123}_{012}+p^{123}_{102}+p^{123}_{112}} = 3 p^{123}_{ij2},
$$
then renormalizes probabilities resulting in $p^{12}_{00} = 0.1, p^{12}_{01} = 0.4, p^{12}_{10} = 0.4, p^{12}_{11} = 0.1$ along with the analogs for $p^{23}_{ij}$ and $p^{13}_{ij}$, which are precisely equivalent to what appears in \ref{fig:inconsistentthreecycle}A as suggested above.

If constraints consistent with those of \ref{fig:inconsistentthreecycle}A are placed on the given network, either the network must add another gene in order to satisfy them directly or the network context imposing those constraints must coarse-grain the network in a suitable way. In what follows, we argue that the former is much more plausible than the latter. This ultimately suggests conditions in which cycle breakage via gene duplication may be selected for to relieve inconsistent constraints that can arise when cycles are present.
