%!TEX root = ../plos_template.tex
% {\noindent\large Significance statement}

% This study addresses the question: Given that networks of stochastically interacting genes determine phenotypes, under what conditions is it possible for natural selection to differentiate among and thus select one gene regulatory network architecture over another? This question arises as a consequence of the significant evidence that transcription is best modeled as a stochastic process. A single gene regulatory network architecture is thus capable of generating many different gene expression patterns, while the same gene expression pattern can be the outcome of many different gene regulatory network architectures.  We demonstrate, even given this ambiguity, a mechanism by which networks possessing cycles may be negatively selected thus biasing the expected distribution over network architectures resulting from the evolutionary process.

% \vskip .7em

{\noindent\large Abstract}

% Different gene regulatory network architectures permit different collections of gene expression patterns. These are defined by the expression states of all genes under consideration, which ultimately determine phenotypes~\cite{Alon2007,Milo2004}. The impact of stochasticity in gene expression~\cite{Eldar2010,Sanchez2013} upon the relationship between each network architecture and its associated potential expression pattern repertoire is not clear~\cite{Jothi2009,Golding2005,Lestas2010,Hilfinger2012,Chalancon2012}. Here we show that gene regulatory networks possessing cycles in their topology, in contrast to those that do not, are capable of having unsatisfiable constraints placed upon them by the network context in which they are embedded. When this occurs, the inconsistency can be resolved by 1) eliminating one of the constraints, 2) duplicating one of the genes in order to break the cycle without modifying the constraints, or 3) having the states of the genes be coarse-grained in terms of their effects on downstream targets. Networks possessing cycles are incapable of satisfying functional requirements that may be imposed upon them by natural selection. This fact may be a significant factor leading to negative selection on network architectures where such inconsistent constraints are relatively likely to arise. In order to investigate this relationship, we take as fundamental the lowest level \gnpm{}: that from gene regulatory networks to gene expression patterns. In order to take stochastic gene expression into account, we mathematically formalize probability distributions over these mappings. This results in the association of a space of probability distributions to each gene regulatory network architecture that defines the latter's functional repertoire. The conclusion then follows from a quantitative comparison of the geometries of spaces of probability distributions on \gnpm{} for different network architectures that determines the likelihood of imposing inconsistent constraints. These results and the associated framework highlight the context-dependence of the functionality of stochastic gene regulatory networks and may also contribute to an explanation for the observation of hierarchical modular gene regulatory network architectures throughout the tree of life~\cite{Ravasz2002,Segre2005,Wagner2007,Erwin2009,Jothi2009,Bhardwaj2010,Colm}.

Constraints placed upon the phenotypes of organisms that result from the interactions of its genes with the environment feed back onto smaller gene networks over evolutionary timescales. The evolution of gene regulatory networks is studied by considering a network of a few genes embedded in a context of other genes and environmental factors.  The gene regulatory network architecture can therefore be defined by the manner in which the gene network is connected to the network context in which it is embedded. Here we show that such network architectures possessing cycles in their topology, in contrast to those that do not, are capable of having unsatisfiable constraints placed upon them. This may be a significant factor leading to negative selection on network architectures where such inconsistent constraints are relatively likely to arise. We quantify the likelihood of inconsistency arising as a function of network architecture and find that networks with a larger number of cycles are more likely to have unsatisfiable constraints placed upon them. Our results highlight the context-dependence of the functionality of gene regulatory networks and may also contribute to an explanation for the observation of hierarchical modular gene regulatory network architectures throughout the tree of life.

