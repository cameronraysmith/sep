\begin{frame}
\begin{footnotesize}
\begin{block}{Future work}
\begin{itemize}
\item compute non-modular:modular ratio for
\begin{enumerate}
\begin{footnotesize}
\item more interesting network topologies on binary graphs
\item graphs with higher-order edges (i.e. hypergraphs)
\end{footnotesize}
\end{enumerate}
\item examine topologies induced on spaces of genotypes, phenotypes and fitnesses by genotype-phenotype and phenotype-fitness mappings
\item design an experiment at the conceptual level to clarify how these ideas can be tested in principle
\item examine relationship to the general \textbf{marginal problem} connecting fields: logic, probability theory, algebraic geometry, machine learning, physics, and perhaps now also biology
\item connect to the algebraic geometric approach to graphical models developed by Bernd Sturmfels (mathematics, UC Berkeley) and Reinhard Laubenbacher (biology, Virgina Tech): \emph{algebraic statistics}
\end{itemize}
\end{block}
\end{footnotesize}
\end{frame}

\begin{frame}
\vspace{3em}
\begin{center}
\includegraphics[width=0.4\textwidth]{fig/ASCB_Cover.jpg}
\end{center}
\end{frame}
