\begin{frame}
\begin{footnotesize}
\begin{block}{Future work}
\begin{itemize}
\item compute non-modular:modular ratio for
\begin{enumerate}
\begin{footnotesize}
\item more interesting network topologies on binary graphs
\item graphs with higher-order edges (i.e. hypergraphs)
\end{footnotesize}
\end{enumerate}
\item examine topologies induced on spaces of genotypes, phenotypes and fitnesses by genotype-phenotype and phenotype-fitness mappings
\item examine relationship to the general \textbf{marginal problem} and continue to connect my approach to those coming from other fields including logic, probability theory, algebraic geometry, machine learning, and physics
\item connect to the algebraic geometric approach to graphical models developed by Bernd Sturmfels (mathematics, UC Berkeley) and Reinhard Laubenbacher (biology, Virgina Tech) in \emph{algebraic statistics} \cite{PachterLior2005}.
\item design an experiment to clarify how these ideas might be tested in principle
\end{itemize}
\end{block}
\end{footnotesize}
\end{frame}

\begin{frame}
\vspace{3em}
\begin{center}
\includegraphics[width=0.4\textwidth]{fig/ASCB_Cover.jpg}
\end{center}
\end{frame}
