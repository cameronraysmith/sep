\begin{frame}
\begin{block}{Conclusions}
\begin{small}
\begin{itemize}
\item For any tree-like (i.e. no cycles in the graphical representation) set of modules, the non-modular (global) and modular (local) spaces of probability distributions on genotype-phenotype mappings have equivalent volume and overlap completely
\item For any collection of modules containing one or more cycles, the space of probability distributions on genotype-phenotype mappings accessed by the modular architecture relative to the non-modular architecture is \emph{larger}.
\item For the n-cycle, as n increases, one approaches a linear chain and the difference between the modular and non-modular spaces of probability distributions shrinks to zero
\item For those cases possessing nested cycles, the difference between the modular and non-modular spaces of probability distributions increases
\end{itemize}
\end{small}
\end{block}
\end{frame}

\begin{frame}
\begin{block}{Conclusions}
\begin{small}
modularity in genotype-phenotype mappings can provide access to a relative superset of possible input-output (genotype-phenotype) relationships. Any environment consistently requiring genotype-phenotype relationships outside the space associated to the non-modular gene regulatory network architecture would induce strong selective power in favor of various modular gene regulatory network architectures having access to these otherwise forbidden states.
\end{small}
\end{block}
\end{frame}
