\begin{frame}
\begin{block}{}
Organisms can be represented as collections of relatively fine-grained \textbf{genes} that are interpreted into phenotypes in the process of gene expression. If there are interaction constraints among the set of genes this representation can be decomposed into relatively coarse-grained \textbf{gene regulatory network modules}. 
\end{block}
\begin{block}{}
Repeated evaluation of the input-output dynamics of genotype-phenotype mappings on a characteristic time-scale induces a \textbf{probability distribution over mappings} from \textbf{collections of genes} to \textbf{collections of phenotypes}.
\end{block}
\begin{block}{}
\textbf{Genotype-phenotype maps} can be evaluated at various levels of organization beginning with a model of the gene and extending from simple molecular to ecosystem-level phenotypes. The level of evaluation determines the content of the collection of \textbf{phenotype values}.
\end{block}
\end{frame}