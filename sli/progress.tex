\begin{frame}
    \begin{block}{Status and Progress since last time}
    \begin{scriptsize}
    \begin{itemize}
    \item Made friends with a category theorist and a couple more mathematicians
    \item Moved toward a clearer thesis question: What are the implications for evolution of combining current knowledge of stochasticity in gene expression and the topological structure of genotype-phenotype mappings?
    \item Current answer: In accordance with the generalized `marginal problem'\footnote{\begin{tiny}
given a list of joint distributions of certain subsets of random variables $A_1, \ldots , A_n$, is it possible to find a joint distribution for all these variables, such that this distribution marginalizes to the given ones? One obvious necessary condition [sometimes referred to as the Kolmogorov consistency conditions] is that for any two of the given distributions which can be marginalized to the same subset of variables, the resulting marginals should be the same \cite{Fritz}
    \end{tiny}} , modularity in genotype-phenotype mappings can provide access to a relative superset of possible input-output (genotype-phenotype) relationships. Any environment consistently requiring genotype-phenotype relationships outside the space associated to the non-modular gene regulatory network architecture would induce strong selective power in favor of various modular gene regulatory network architectures having access to these otherwise forbidden states.
    \item Data analysis: published 3 papers, worked on 2 more in collaboration with the Edelmann and Casadevall labs; 1 previous paper in collaboration with the Skoultchi lab
    \end{itemize}
    \end{scriptsize}
    \end{block}
    \end{frame}
