\begin{frame}
In the first case, $\mathcal{E}(L) = P^L$ we have $\{t_1, \ldots t_m\} = \{t_j | t_j \in P^L\}$. In the second case, $\coprod_{O \in \mathcal{G}} \mathcal{E}(O)$, we have $\{e_1, \ldots e_n\} = \coprod_{O \in \mathcal{G}} \mathcal{E}(O)$. So we have two sets of maps, one defined on $P^L$ and the other defined on $\mathcal{E}(O) = P^O$ for each $O \in \mathcal{G}$. This yields a method of specifying the intended relationship that defines $\mathbf{G}_{n \times m}$ in terms of $t_j \in P^L$ and $e_i \in \mathcal{E}(O)$ :
\begin{eqnarray*}
\mathbf{G}[i,j] =
\begin{cases}
1, & t_j|O = e_i,\\
0, & \text{otherwise}.
\end{cases}
\end{eqnarray*}
\end{frame}
