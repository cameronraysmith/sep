\begin{frame}
Probability simplices and constraints placed upon them by linear transformations like G give rise to polytopes
\begin{block}{\textbf{Polytope}}
\textbf{Convex polytopes} can be represented in two different but equivalent forms. The so-called \textbf{H-representation} in terms of the set of solutions of a system of linear inequalities
$$K\mathbf{x} \geq \mathbf{b}$$
or the so-called \textbf{V-representation} in terms of the convex hull
$$\left\{ \mathbf{x} \in \mathbb{R}^d | \mathbf{x} = \sum_{i=1}^n \lambda_i \mathbf{x_i}, \lambda_i \in \mathbb{R}, \lambda_i \geq 0, \sum_{i=1}^n \lambda_i = 1  \right\} $$
of a finite set of points.
\end{block}
\end{frame}
