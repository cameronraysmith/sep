% A simple graph with straight and bend arrows and loops
% Stefan Kottwitz
\documentclass{article}
\usepackage{tikz}

%%%<
\usepackage{verbatim}
\usepackage[active,tightpage]{preview}
\PreviewEnvironment{tikzpicture}
\setlength\PreviewBorder{5pt}%
%%%>
\begin{comment}
:Title: Simple graph
:Tags: Arrows;Diagrams;Graphs;Mathematics
:Author: Stefan Kottwitz
:Slug: graph

A simple example of a graph with straight and bend arrows and loops.
It has been posted as answer to the question
http://tex.stackexchange.com/q/45734/213 of Ichibann.

* Define styles for edges, arrows, and nodes
* Place the main nodes
* Draw edges with nodes for description
* Use options `loop` and `bend` for loops and bent edges
* Specify `left` and `right` for bend direction and node placement
\end{comment}
\usetikzlibrary{arrows}
\begin{document}

\begin{tikzpicture}[->,>=stealth',shorten >=1pt,auto,node distance=3cm,
  thick,main node/.style={circle,fill=white,draw=wisteria,font=\sffamily}]

  \definecolor{nephritis}{RGB}{39, 174, 96} % green
  \definecolor{wisteria}{RGB}{142, 68, 173} % purple
  \definecolor{pumpkin}{RGB}{211, 84, 0} % orange
  \definecolor{belizehole}{RGB}{41, 128, 185} % blue

  \node[main node] (top) {$\emptyset$};
  \node[main node] (B) [below left of=top] {B};
  \node[main node] (A) [left of=B] {A};
  \node[main node] (C) [below right of=top] {C};
  \node[main node] (D) [right of=C] {D};
  \node[main node] (AB) [below of=A] {AB};
  % \node[main node] (AC) [right of=AB] {AC};
  \node[main node] (AD) [below of=B] {AD};
  \node[main node] (BC) [below of=C] {BC};
  \node[main node] (BD) [below of=D] {BD};
  % \node[main node] (CD) [right of=BD] {CD};

  \path[every node/.style={font=\sffamily\small}]
    (top) edge node [left] {} (A)
        edge node [left] {} (B)
        edge node [left] {} (C)
        edge node [left] {} (D)
    (A) edge node [left] {} (AB)
        % edge node [left] {} (AC)
        edge node [left] {} (AD)
    (B) edge node [left] {} (AB)
        edge node [left] {} (BC)
        edge node [left] {} (BD)
    (C) edge node [left] {} (BC)
        % edge node [left] {} (CD)
        % edge node [left] {} (AC)
    (D) edge node [left] {} (AD)
        edge node [left] {} (BD)
        % edge node [left] {} (CD)
    ;
\end{tikzpicture}
\end{document}
